\documentclass[12pt]{article}
\usepackage[utf8]{inputenc}
\usepackage{amsmath}
\usepackage{graphicx}
\usepackage{hyperref}
\usepackage{listings}

\title{Izvještaj: Laboratorijska vježba za skrivene Markovljeve modele}
\author{Dominik Barukčić}
\date{\today}

\begin{document}
	
	\maketitle
	
	\section{Uvod}
	Kratki uvod o laboratorijskoj vježbi i osnovama skrivenih Markovljevih modela.
	
	\section{Opis Eksperimenta}
	Detaljan opis stohastičkog eksperimenta s pristranim kockama.
	
	\section{Pod-zadaci}
	\subsection{Pod-zadatak 1: Definiranje HMM Modela}
	Ovdje opišite kako ste definirali HMM model, uključujući inicijalizaciju stanja i matricu prijelaza.
	
	\subsection{Pod-zadatak 2: Log-Izvjesnosti Osmotrenih Nizova}
	Detalji o log-izvjesnostima za zadane osmotrene nizove.
	
	\subsection{Pod-zadatak 3: Vjerojatnosti Unaprijed i Unazad}
	Opišite kako ste izračunali alfa i beta vjerojatnosti.
	
	\subsection{Pod-zadatak 4: Dekodiranje Skrivenih Stanja}
	Detalji o korištenju Viterbi algoritma za odredivanje najizvjesnijeg niza skrivenih stanja.
	
	\subsection{Pod-zadatak 5: Log-Izvjesnosti duž Viterbi Puta}
	Usporedba log-izvjesnosti duž Viterbi puta s ukupnom log-izvjesnošću.
	
	\subsection{Pod-zadatak 6: Izvjesnost Osmotrenih Nizova za Skraćeni Niz}
	Analiza izvjesnosti osmotrenih nizova za skraćene sekvence.
	
	\subsection{Pod-zadatak 7: Generiranje Osmotrenih Nizova}
	Opis procesa generiranja osmotrenih nizova.
	
	\subsection{Pod-zadatak 8: Dugotrajne Statistike i Teorijska Očekivanja}
	Analiza dugotrajnih statistika osmotrenih simbola.
	
	\subsection{Pod-zadatak 9: Log-Izvjesnost Generiranih Osmotrenih Nizova}
	Razmatranje log-izvjesnosti generiranih nizova.
	
	\subsection{Pod-zadatak 10: Treniranje Parametara HMM Modela}
	Detalji o procesu treniranja parametara HMM modela.
	
	\subsection{Pod-zadatak 11: Usporedna Evaluacija Modela}
	Usporedba različitih modela temeljena na log-izvjesnosti.
	
	\section{Zaključak}
	Vaši generalni dojmovi, zaključci i uvidi dobiveni tijekom laboratorijske vježbe.
	
	\section{Prilozi}
	Ovdje dodajte sve dodatne materijale, kodove, grafikone itd.
	
\end{document}
